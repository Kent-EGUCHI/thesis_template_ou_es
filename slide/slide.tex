\documentclass[12pt,dvipdfmx,svgnames,uplatex,aspectratio=169]{beamer}
% \documentclass[12pt,dvipdfmx,svgnames,uplatex,aspectratio=169,draft]{beamer}  % 画像を省略してコンパイルを高速化する
%
% ===========================================
% 図・表関係
% ===========================================
\usepackage{graphicx}  % documentclassのオプションに draft を追加 -> ダミー図となりコンパイルが早くなる
\graphicspath{{../thesis/pics/}}  % \includegraphicsで参照するディレクトリ
%これを \graphicspath{{../thesis/pics/}, {/path/to/figure/directory/}} のように書けば複数のディレクトリの画像を参照できる.
\usepackage{graphbox}
\usepackage[font=small,labelfont=bf]{caption}
%
% ===========================================
% 参考文献
% ===========================================
\usepackage[url=false,isbn=false,doi=false]{biblatex}
\addbibresource{../bib_textbooks.bib}  % 教科書など
\addbibresource{../bib_articles.bib}  % 論文など
%
% ===========================================
% 独自スタイルの導入
% ===========================================
\usepackage{./mystyle_slide}
\usepackage{pifont}% http://ctan.org/pkg/pifont
\newcommand{\cmark}{\ding{51}}%
\newcommand{\xmark}{\ding{55}}%
\usepackage{cancel}
%
% ===========================================
% 表紙の記述
% ===========================================
\title{
  論文のタイトル論文のタイトル
}
\subtitle{
  20〇〇年度~大阪大学大学院基礎工学研究科\\
  機能創成専攻~修士論文公開講演会
}
\author{阪大~太郎}
\institute{〇〇研究室}
\date{20〇〇年2月〇〇日}

% +++++++++++++++++++++++++++++++++++++++++++
% 本文
% +++++++++++++++++++++++++++++++++++++++++++

\begin{document}
\thispagestyle{empty}
\maketitle

% ===========================================
% 目次
% ===========================================
\begin{frame}[noframenumbering]{目次}
  \thispagestyle{empty}
  % \tableofcontents[hideallsubsections]  % 一列,節のみ
  \tableofcontents  % 一列,小節まで
  % 二列,小節まで
  % \begin{columns}[t]
  %   \begin{column}{.5\textwidth}
  %     \tableofcontents[sections={1-3}]
  %   \end{column}
  %   \begin{column}{.5\textwidth}
  %     \tableofcontents[sections={4-7}]
  %   \end{column}
  % \end{columns}
\end{frame}


\addtocounter{framenumber}{-1}  % 次のスライドからページカウントを開始する
% ===========================================
% 研究背景
% ===========================================
\section{研究背景}
\subsection{乱流現象}
\begin{frame}{\insertsection:\insertsubsection}
  \begin{columns}[c]
    \begin{column}{0.4\textwidth}
      \centering
      \includegraphics[width=\textwidth]{example-image-1x1}
    \end{column}
    \begin{column}{0.6\textwidth}
      \begin{block}{乱流}
        \begin{itemize}
          \item 流体が時空間的に乱れたふるまいを示す
          \item 決定論的な支配方程式に従うが,\\
            その非線形性から解析が困難
        \end{itemize}
      \end{block}
      \pause
      \begin{block}{エネルギカスケード機構}
        \begin{enumerate}
          \item 大スケールへのエネルギ注入
          \item 大 \(\to\) 小スケールへのエネルギ伝達
          \item 小スケールからのエネルギ散逸
        \end{enumerate}
      \end{block}
    \end{column}
  \end{columns}
  \vspace{\baselineskip}
  \pause
  \hfill
  \begin{beamercolorbox}[rounded=true, wd=0.8\textwidth]{stress}
    \centering
    「乱流の動力学」はきわめて重要な問題だが,\\その完全な性質は明らかになっていない
  \end{beamercolorbox}
  \hfill\hfill
\end{frame}


\subsection{乱流の普遍則}
\begin{frame}{\insertsection:\insertsubsection}
  \begin{columns}[c]
    \begin{column}{0.6\textwidth}
      \begin{block}{相似則}
        小スケールの統計は平均エネルギ散逸率\(\overline{\epsilon}\)と波数スケール\(k\)で決まる:
        \begin{equation*}
          E(k) = C \overline{\epsilon}^{2/3} k^{-5/3}
        \end{equation*}
      \end{block}
      \uncover<2>{
      \begin{block}{散逸則}
        \(\overline{\epsilon}(t)\)を大スケールの特徴速度\(U(t)\)と特徴長さ\(L(t)\)で評価できる:
        \begin{equation*}
          \overline{\epsilon}(t) = \overline{\varPi_L}(t) ~\propto ~\frac{U(t)^3}{L(t)}
        \end{equation*}
      \end{block}
      }
    \end{column}
    \begin{column}{0.4\textwidth}
      \centering
      \only<1>{
      \includegraphics[width=\textwidth]{example-image-a}
      }
      \only<2>{
      \includegraphics[width=\textwidth]{example-image-b}
      }
    \end{column}
  \end{columns}
\end{frame}


% ===========================================
% 研究目標
% ===========================================
\section{研究目標}
\begin{frame}{\insertsection}
  \hfill
  \begin{beamercolorbox}[rounded=true, wd=0.8\textwidth]{stress}
    \centering
    研究目標研究目標研究目標研究目標研究目標研究目標
  \end{beamercolorbox}
  \hfill\hfill
  % \vspace{\baselineskip}
  \pause
  \begin{columns}[T]
    \begin{column}{0.4\textwidth}
      \vspace{\baselineskip}
      \centering
      \includegraphics[width=\textwidth]{example-image-1x1}
    \end{column}
    \begin{column}{0.6\textwidth}
      \begin{block}{対象とする系}
        \begin{itemize}
          \item 対象とする系の説明
          \item 対象とする系の説明
          \item 対象とする系の説明
          \item 対象とする系の説明
          \item 対象とする系の説明
          \item 対象とする系の説明
          \item 対象とする系の説明
        \end{itemize}
      \end{block}
    \end{column}
  \end{columns}
\end{frame}


\section{解析手法}
\subsection{室内実験}
\begin{frame}{\insertsection:\insertsubsection}
  \begin{columns}[c]
    \begin{column}{0.3\textwidth}
      \centering
      \includegraphics[width=\textwidth]{example-image-9x16}
    \end{column}
    \begin{column}{0.7\textwidth}
      \begin{block}{実験装置}
        \begin{table}
          \begin{tabular}{l|l}
            系の直径  & \(D=100\, \si{mm}\) \\
            系の高さ  & \(H=300\, \si{mm}\) \\
            系の周期  & \(T=0.1 \sim 10\, \si{s}\)
          \end{tabular}
        \end{table}
      \end{block}
      \pause
      \begin{block}{実験条件}
        \vspace{-\baselineskip}
        \begin{table}
          \begin{tabular}{l|cc}
            & 条件A & 条件B \\ \hline
            \makebox[5zw][l]{測定領域} \(A\) & \((D/2)^2\) & \((D/2)^2\) \\
            空間解像度 \(\Delta x\) & \(0.01D\) & \(0.02D\) \\
            時間解像度 \(\Delta t\) & \(1.0\times10^{-2}T\) & \(5.0\times10^{-3}T\) \\
            \makebox[5zw][l]{測定時間} \(T_\mathrm{max}\) & \(10T\) & \(500T\)
          \end{tabular}
        \end{table}
      \end{block}
    \end{column}
  \end{columns}
\end{frame}


% ===========================================
% 結果1
% ===========================================
\section{結果1}
\subsection{大スケール構造と秩序構造}
\begin{frame}{\insertsection:\insertsubsection}
  \begin{columns}[T]
    \begin{column}{0.33\textwidth}
      \begin{block}{流速場}
        \centering
        \includegraphics[width=\textwidth]{example-image-1x1}
        \begin{equation*}
          \vb*{u}(\vb*{x}, t)
        \end{equation*}
      \end{block}
    \end{column}
    \begin{column}{0.33\textwidth}
      \begin{block}{自乗ストレイン場}
        \centering
        \includegraphics[width=\textwidth]{example-image-1x1}
        \begin{equation*}
          S(\vb*{x}, t)
        \end{equation*}
      \end{block}
    \end{column}
    \begin{column}{0.33\textwidth}
      \begin{block}{エンストロフィ場}
        \centering
        \includegraphics[width=\textwidth]{example-image-1x1}
        \begin{equation*}
          \varOmega(\vb*{x}, t)
        \end{equation*}
      \end{block}
    \end{column}
  \end{columns}
  \begin{itemize}
    \item 大スケール構造の逆相関\(\to\)\colorbox{pink}{秩序的な構造}
  \end{itemize}
\end{frame}


% ===========================================
% 結果2
% ===========================================
\section{結果2}
\subsection{エネルギの自己相関}
\begin{frame}{\insertsection:\insertsubsection}
  \centering
  \includegraphics[width=0.7\textwidth]{example-image-16x9}
  \begin{itemize}
    \item 自己相関を \(C_0\exp[-\tau/\tau_0]\) でフィッティング
    \item 普遍的な時定数 \(\tau_0=12\) \(\to\) 三桁の \(\Re\) で\colorbox{pink}{\(\order{10T}\) の大スケール運動}
  \end{itemize}
\end{frame}


% ===========================================
% 結言
% ===========================================
\section{結言}
\begin{frame}{\insertsection}
  \hfill
  \begin{beamercolorbox}[rounded=true, wd=0.8\textwidth]{stress}
    \centering
    閉じた系の乱流の時間変動と空間構造を解析
  \end{beamercolorbox}
  \hfill\hfill
  \vspace{\baselineskip}
  \begin{columns}[T]
    \begin{column}{0.3\textwidth}
      \centering
      \includegraphics[height=0.5\textheight]{example-image-9x16}
    \end{column}
    \begin{column}{0.7\textwidth}
      \begin{block}{空間間欠性}
        \vspace{-0.5\baselineskip}
        \begin{itemize}
          \item[\cmark] 小スケール構造のクラスタ(PIV)
          \item[\cmark] 逆相関する大スケール構造(PIV)
        \end{itemize}
      \end{block}
      \begin{block}{時間非定常性}
        \vspace{-0.5\baselineskip}
        \begin{itemize}
          \item[\cmark] \(\order{10T}\) の長時間相関(PIV)
        \end{itemize}
      \end{block}
      \begin{block}{非平衡なカスケード}
        \vspace{-0.5\baselineskip}
        \begin{itemize}
          \item[\cmark] 主流と二次流の間の時間遅れ(DNS)
        \end{itemize}
      \end{block}
    \end{column}
  \end{columns}
  \pause
  \hfill
  \begin{beamercolorbox}[rounded=true, wd=0.8\textwidth]{stress}
    \centering
    局所平衡仮説の破れを示唆する結果\\
    \(\to\) 動力学に基づいた新理論の基礎
  \end{beamercolorbox}
  \hfill\hfill
\end{frame}



% ===========================================
% 補足スライド
% ===========================================
\backupbegin
\appendix

\begin{frame}
  \centering
  \begin{beamercolorbox}[wd=\textwidth, center, sep=2pt, rounded=true, shadow=false]{frametitle}
    補足スライド
  \end{beamercolorbox}
\end{frame}


\section{補足スライド}
\subsection{補足トピック1}
\begin{frame}{\insertsection:\insertsubsection}
  \begin{columns}[T]
    \begin{column}{0.33\textwidth}
      \begin{block}{\(f(\vb*{x},t)\)}
        \centering
        \includegraphics[width=\textwidth]{example-image-1x1}
      \end{block}
    \end{column}
    \begin{column}{0.33\textwidth}
      \begin{block}{\(g(\vb*{x},t)\)}
        \centering
        \includegraphics[width=\textwidth]{example-image-1x1}
      \end{block}
    \end{column}
    \begin{column}{0.33\textwidth}
      \begin{block}{\(h(\vb*{x},t)\)}
        \centering
        \includegraphics[width=\textwidth]{example-image-1x1}
      \end{block}
    \end{column}
  \end{columns}

  \begin{itemize}
    \item 補足情報1
    \item 補足情報2
  \end{itemize}
\end{frame}

\backupend

\end{document}
